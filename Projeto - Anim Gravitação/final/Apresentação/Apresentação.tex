\documentclass[compress]{beamer}
%\documentclass[ignorenonframetext,handout]{beamer}
%\setbeamercovered{transparent}
%\usepackage[ISO 8859-1]{inputenc}
%\usepackage{default}

% para usar figuras devemos acrescentar
\usepackage{graphicx}
%\usepackage{graphics}
%\DeclareGraphicsExtensions{.pdf,.png,.jpg}
%\DeclareGraphicsExtensions{.jpg, .eps}
%\DeclareGraphicsRule{.jpg}{eps}{.jpg}{`jpeg2ps -h -r 600 #1}
%\usepackage{tikz}
%\usepackage{bm}
%\usetikzlibrary{arrows,backgrounds,coordinatesystems,3d,shapes,plotmarks,automata,calendar,er,
%folding,matrix,mindmap,patterns,petri,plothandlers,topaths,trees} 
%\usetikzlibrary{positioning}
%\usepgflibrary{decorations.pathreplacing}
%\usetikzlibrary{decorations.pathreplacing}
%\usetikzlibrary{decorations.pathmorphing}
%\usetikzlibrary[patterns]
%\tikzstyle{every text node part}
%\usetikzlibrary{arrows,backgrounds,positioning,fit} 
%\usetikzlibrary{calc}
% para gerar graficos no latex
%\usepackage{pgfplots}
%\pgfplotsset{compat=newest}

\usepackage{amsfonts}
\usepackage{amssymb}
\usepackage{amsmath}
\usepackage{MnSymbol}

\usepackage[brazil]{babel}
%\usepackage[utf8]{inputenc}

% \usepackage{algpseudocode}
% \usepackage{algorithmicx}
\usepackage[Algoritmo]{algorithm}
\usepackage[noend]{algorithmic}

\setbeamertemplate{bibliography entry title}{}
\setbeamertemplate{bibliography entry location}{}
\setbeamertemplate{bibliography entry note}{}

\newcounter{saveenumi}
\newcommand{\seti}{\setcounter{saveenumi}{\value{enumi}}}
\newcommand{\conti}{\setcounter{enumi}{\value{saveenumi}}}

%\usepackage{shadethm}

%\definecolor{shadethmcolor}{rgb}{.75,.75,.75}

%\newshadetheorem{theorem}{\scshape Teorema}[chapter]
\newtheorem{teorema}[theorem]{\scshape Teorema}
\newtheorem{proposicao}[theorem]{\scshape Proposição}
\newtheorem{corolario}[theorem]{\scshape Corolário}
\newtheorem{lema}[theorem]{\scshape Lema}
\newtheorem{definicao}[theorem]{\scshape Definição}
\newtheorem{conjectura}[theorem]{\scshape Conjectura}
\newtheorem{escolio}[theorem]{\scshape Escólio}
\newtheorem{exemplo}[theorem]{\scshape Exemplo}
\newtheorem{exemplos}[theorem]{\scshape Exemplos}
\newtheorem{propriedade}[theorem]{\scshape Propriedade}

\renewcommand{\u}{{\bf u}}
\renewcommand{\v}{{\bf v}}
\renewcommand{\sin}{\operatorname{sen}}
\providecommand{\cas}{\operatorname{cas}}
\providecommand{\mdc}{\mathrm{mdc}}
\providecommand{\f}{{\bf f}}

\newcommand{\ie}{\textit{i.e.}}
\newcommand{\eg}{\textit{e.g.}}
%\newcommand{\qed}{\hfill $\square$}

\renewcommand\Re{\operatorname{Re}}
\renewcommand\Im{\operatorname{Im}}

\providecommand{\x}{{\bf x}}
\providecommand{\y}{{\bf y}}
\providecommand{\w}{{\bf w}}
\providecommand{\f}{{\bf f}}
\providecommand{\q}{{\bf q}}
\providecommand{\bfa}{{\bf a}}
\providecommand{\bfb}{{\bf b}}
\providecommand{\bfc}{{\bf c}}
\providecommand{\bfd}{{\bf d}}
\providecommand{\bfe}{{\bf e}}
\providecommand{\bfs}{{\bf s}}
\providecommand{\bfz}{{\bf z}}
\providecommand{\zero}{{\bf 0}}
\providecommand{\spn}{\mathrm{span}}
\providecommand{\posto}{\mathrm{posto}}
\providecommand{\nul}{\mathrm{nul}}
\providecommand{\proj}{\mathrm{proj}}
\providecommand{\tr}{\mathrm{tr}}
\providecommand{\sgn}{\mathrm{sgn}}

\providecommand{\cov}{\mathrm{cov}}

\providecommand{\dilation}{\mathcal{D}}
\providecommand{\erosion}{\mathcal{E}}
\providecommand{\open}{\mathcal{O}}
\providecommand{\close}{\mathcal{C}}

\newcommand*{\Bhat}{\skew{3}{\hat}{B}}

\mode<presentation>
{
  \setbeamertemplate{background canvas}[vertical shading][bottom=white!10,top=blue!10]
%  \usetheme{Berkeley}
  \usetheme{CambridgeUS}
%  \usetheme{Madrid}
%  \usetheme{Warsaw}
  \usefonttheme[onlysmall]{structurebold}
  
  \setbeamertemplate{headline}{}
  
%   \setbeamercovered{invisible} % default
  \setbeamercovered{ transparent, again covered={\opaqueness{25}} } % =15%
%   \setbeamercovered{transparent=50}
%   \setbeamercovered{dynamic}

%   \setbeamercovered{again covered={\opaqueness<1->{25}}}
}

% copiado do site:
% http://latex-beamer-class.10966.n7.nabble.com/Covering-images-transparent-i-e-dimmed-figures-td1504
% . html
\usepackage{ifthen}

\makeatletter
\newcommand{\includecoveredgraphics}[2][]{
    \ifthenelse{\the\beamer@coveringdepth=1}{
        \tikz
            \node[inner sep=0pt,outer sep=0pt,opacity=0.15]
                {\includegraphics[#1]{#2}};
    }{
        \tikz
            \node[inner sep=0pt,outer sep=0pt]
                {\includegraphics[#1]{#2}};%
    }
} 
\makeatother % não sei se precisa...


% para a disciplina de Processamento de Imagens
\title{Interação Gravitacional Entre Três Corpos}
\author{Alice Amaral\\ José Ribeiro\\ Marcos Pereira}
\date{UnB - Universidade de Brasília\\
IFD - Instituto de Física\\
Métodos Computacionais A - 1º/2018}
\institute{}



\begin{document}


\frame{\titlepage}

%%% SUMÁRIO %%%
\section{Sumário}
%\frame{\tableofcontents}
%\section{}

\begin{frame}{Sumário}
\begin{enumerate}
\item<+->{Teoria}
\item<+->{Algoritmo}
\item<+->{Programa}
\item<+->{Resultados}
\item<+->{Animações}
\end{enumerate}
\end{frame}
%\begin{frame}{\tableofcontents}
%\end{frame}

%%%%%%%%%%%%%%%%%%%%%%%%%%%%%%

\begin{frame}{Abordagem Teórica}
\section{Abordagem Teórica}
\subsection{Interação Gravitacional Entre Dois Corpos}
Potencial gravitacional
\begin{equation}
    \varPhi_{1}=-G\frac{m_{1}}{\left[\left(x_2-x_1\right)^2+\left(y_2-y_1\right)^2\right]^{\frac{1}{2}}}
\end{equation}
Energia potencial
\begin{equation}
U_{12}=m_{2}\varPhi_{1}    
\end{equation}
logo, se $$\mathbf{F}_{ij}=-\bm{\nabla}U_{ij}=-m_j\bm{\nabla}\varPhi_i$$
\begin{center}
\begin{displaymath}
-\bm{\nabla}U_{12}=-G\frac{m_1m_2\mathbf{r}_{12}}{\left[\left(x_2-x_1\right)^2+\left(y_2-y_1\right)^2\right]^{\frac{3}{2}}}=\bm{\nabla}U_{21}
\end{displaymath}

\end{center}

\end{frame}


%\begin{frame}
%    \begin{figure}[h!]
%        \centering
%        \includegraphics[scale=.6]{fig1.png}
%        \caption{Sistema Sol-Terra}
%        \label{fig:solterra}
%    \end{figure}
%\end{frame}
%%%
\begin{frame}{Interação Gravitacional Entre Três Corpos}
\uncover<+->{Os três corpos interagem entre si perturbando as trajetórias:}


\begin{itemize}
\vfill \item<+->{Interação entre 1 e 2:
\begin{equation*}
 \mathbf{F}_{12}=-G\frac{\mathbf{r}_{12}}{\left[\left(x_2-x_1\right)^2+\left(y_2-y_1\right)^2\right]^{\frac{3}{2}}}=-\mathbf{F}_{21} \, .
\end{equation*}
}

\vfill \item<+->{Interação entre 2 e 3:
\begin{equation*}
  \mathbf{F}_{32}=-G\frac{\mathbf{r}_{32}}{\left[\left(x_3-x_2\right)^2+\left(y_3-y_2\right)^2\right]^{\frac{3}{2}}}=-\mathbf{F}_{23} \, .
\end{equation*}
}

\vfill \item<+->{Interação entre 1 e 3:
\begin{equation*}
 \mathbf{F}_{13}=-G\frac{\mathbf{r}_{13}}{\left[\left(x_3-x_1\right)^2+\left(y_3-y_1\right)^2\right]^{\frac{3}{2}}}=-\mathbf{F}_{31} \, . 
\end{equation*}
}

\end{itemize}
\end{frame}

\begin{frame}{Matriz de co-ocorrência}
\uncover<+->{Matriz de co-ocorrência com $d = 1$ e $\theta = 0^{\circ}$:}

%%begin novalidate
\vfill
\uncover<+->{
\begin{center}
\begin{tabular}{|c|c|c|c|c|}
\hline
3 & 2 & 0 & 1 & 0  \\
\hline
1 & 2 & 1 & 3 & 0  \\
\hline
3 & 1 & 0 & 2 & 3  \\
\hline
1 & 2 & 3 & 0 & 3  \\
\hline
0 & 0 & 0 & 0 & 1 \\
\hline
\end{tabular}
}
\uncover<+->{
\quad
\begin{tabular}{c|c c c c}
  & 0 & 1 & 2 & 3  \\
\hline
0 & 3 & 2 & 1 & 1  \\

1 & 2 & 0 & 2 & 1  \\

2 & 1 & 1 & 0 & 2  \\

3 & 2 & 1 & 1 & 0 \\
\end{tabular}
\end{center}
}
%%end novalidate

\vfill
\uncover<+->{
\begin{center}
\begin{tabular}{c|c c c c}
  & 0 & 1 & 2 & 3  \\
\hline
0 & 0.15 & 0.10 & 0.05 & 0.05  \\

1 & 0.10 & 0.00 & 0.10 & 0.05  \\

2 & 0.05 & 0.05 & 0.00 & 0.10  \\

3 & 0.10 & 0.05 & 0.05 & 0.00 \\
\end{tabular}
\end{center}
}
\end{frame}




\begin{frame}{Matriz de co-ocorrência}
\uncover<+->{
\begin{eqnarray*}
 P(i,j,d,135^{\circ}) &=& \#\{\{(k,l),(m,n)\} \subset S \, | \, (k - m = d, l - n = d) \mbox{ ou}\\
 && (k - m = -d, l - n = -d), f(k,l) = i, f(m,n) = j \} \, .
\end{eqnarray*}
}

\vfill
\uncover<+->{Pode-se normalizar os elementos da matriz de co-ocorrência para que representem
probabilidades fazendo:}

\vfill
\uncover<+->{
\begin{equation*}
 p(i,j) = \frac{P(i,j)}{\sum\limits_{p = 0}^{H_g} \sum\limits_{q = 0}^{H_g} P(p,q)} \, , 
\end{equation*}
}

\uncover<+->{sendo $H_g$ o nível de cinza máximo na imagem.}
\end{frame}

\begin{frame}{Tipos de Sinais -- Exemplos}
\uncover<+->{
Uma imagem pode ser vista como um sinal que é, por sua vez, uma função de duas variáveis espaciais:
$x$ e $y$.}


%\uncover<+->{
%%\begin{figure}
%% \includecoveredgraphics[width=0.9\paperwidth, clip=true, trim=50mm 100mm 10mm
%%0]{./figures/fig_image_signal.jpg}
%%\end{figure}

\end{frame}
'




%%% BIBLIOGRAFIA %%%
\begin{frame}[allowframebreaks]
\frametitle{Referências}
\bibliographystyle{plain}
\bibliography{biblio.bib}
\end{frame}



\end{document}
